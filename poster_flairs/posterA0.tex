\documentclass[final,t]{beamer}
\mode<presentation>{ \usetheme{Major} }
\usepackage{times}
\usepackage{amsmath,amsthm, amssymb, latexsym}
\boldmath
\usepackage[english]{babel}

\usepackage{relsize}
\usepackage{multirow}
\usepackage{qtree}
\usepackage{stmaryrd}
\usepackage{booktabs}

%  \usepackage[font=small,format=plain,labelfont=bf,up,textfont=it,up]{caption}
\usepackage[font=normalsize,labelfont=small,bf,margin=2cm]{caption}
\usepackage[latin1]{inputenc}
\usepackage[orientation=landscape,size=a0,scale=1.4,debug]{beamerposter}
\usepackage{color,listings}
\usepackage{calc,xcolor}
\usepackage[absolute,overlay]{textpos}
\usepackage{smartdiagram}

\usepackage{wrapfig}

\usepackage[many]{tcolorbox}
\usepackage{tikz}
\usetikzlibrary{shadings}

\definecolor{lightblue}{rgb}{.85,.85,1} % 217 217 255
\definecolor{lightorange}{rgb}{1,.8,.6} % 255 204 153
\definecolor{lightgreen}{rgb}{.77,.91,.5} % 196 232 128
\definecolor{lightyellow}{rgb}{1,.98,.6} % 255 250 153
\definecolor{prussianblue}{rgb}{0.0,0.19,0.33}
\definecolor{gold}{rgb}{0.6,0.4,0.08}

\makeatletter
\pgfdeclarehorizontalshading{beamer@headfade}{5.4375ex+490pt}
{%
  color(0cm)=(lightblue);
  % color(15cm)=(lightblue);
  color(45cm)=(lightyellow);
  color(\paperwidth)=(gold)%
}
\addtoheadtemplate{\pgfuseshading{beamer@headfade}\vskip\dimexpr 0pt-52ex}{}
\makeatother


\newsavebox\CBox
\newenvironment{ColorBox}[3][black]{
    \par\noindent
    \def\borderColor{#1}\def\bgColor{#2}
    \begin{lrbox}{\CBox}
    \minipage{#3-2\fboxsep-2\fboxrule}
}{
    \endminipage\end{lrbox}%
    \fcolorbox{\borderColor}{\bgColor}{\usebox\CBox}\par
}

\lstset{language=java}
\lstset{breaklines=true}
\lstset{showstringspaces=false}
\lstset{tabsize=3}
\lstset{basicstyle=\ttfamily\scriptsize}
\lstset{breakautoindent=true}
\lstset{postbreak=\space}
%\lstset{commentstyle=\color{XcodeComments}}
%\lstset{keywordstyle=\color{XcodeKeywords}}
%\lstset{stringstyle=\color{XcodeStringstyle}}

%%%%%%%%%%%%%%%%%%%%%%%%%%%%%%%%%%%%%%%%%%%%%%%%%%%%%%%%%%%%%%%%%%%%%%%%%%%%%%%%%5
\title[]{An Underwater Robotic Smart-Sensing System for Water Quality Testing }
\author[Wright]{Elisia Wright, David Boughton and Dr. Janyl Jumadinova}
\institute{Department of Computer Science, Allegheny College \\ Meadville, PA}

%%%%%%%%%%%%%%%%%%%%%%%%%%%%%%%%%%%%%%%%%%%%%%%%%%%%%%%%%%%%%%%%%%%%%%%%%%%%%%%%%5
\begin{document}
    \begin{frame}{}
       % \vspace*{-6mm}
        \begin{columns}[t]
        	\begin{column}{1\linewidth}
            %%%%%%%%%%%%%%%%%%%%%%%%%%%%%%%%%%%%%%%%%%%%%%%%%%%%%%%%%%%%%%%%%%%%%
            %
            % Center column - Context
            %
            %%%%%%%%%%%%%%%%%%%%%%%%%%%%%%%%%%%%%%%%%%%%%%%%%%%%%%%%%%%%%%%%%%%%%

                %%%%%%%%%%%%%%%%%%%%%%%%%%%%%%%%%%%
                %
                % Project Objectives
                %
                %%%%%%%%%%%%%%%%%%%%%%%%%%%%%%%%%%%
                \begin{alertblock}{\textsc{\textbf{Project Objectives}}}
                    \vspace*{3mm}
                    The current methods for water quality testing either use a
                    single sensor to get a  sample for testing each water
                    quality parameter or data buoys that are able to
                    obtain readings from multiple sensors at a stationary location.
                    \vspace{3mm}

                    \emph{This project presents:}
                    \begin{itemize}
                        \item A single unit comprised of \textbf{multiple sensors}
                        that are able to collect data simultaneously for water
                        quality testing.
                        \item This multi-sensor unit attachable to the \textbf{underwater robot}
                        to collect data at various depths of the water column for
                        an extended period of time.
                        \item \textbf{Data collection} and \textbf{data analysis}
                        software to manage the data and assess certain trends in
                        the water quality over time.
                    \end{itemize}
                    \vspace*{6mm}
                \end{alertblock}
			\end{column}
		\end{columns}

		%%%%%%%%%%%%%%%%%%%%%%%%%%%%%%%%%%%%%%%%%%%%%%%%%%%%%%%%%%%%%%%%%%%%%
		\begin{columns}
		    %%%%%%%%%%%%%%%%%%%%%%%%%%%%%%%%%%%%%%%%%%%%%%%%%%%%%%%%%%%%%%%%%%%%%
            %
            % Left column - Context
            %
            %%%%%%%%%%%%%%%%%%%%%%%%%%%%%%%%%%%%%%%%%%%%%%%%%%%%%%%%%%%%%%%%%%%%%
            \begin{column}{.33\linewidth}
                %%%%%%%%%%%%%%%%%%%%%%%%%%%%%%%%%%%
                %
                % Method
                %
                %%%%%%%%%%%%%%%%%%%%%%%%%%%%%%%%%%%
                \begin{block}{\textsc{\textbf{Algal Blooms}}}
                  \vspace*{3mm}
                       Lake Erie algal blooms are an annual threat to the health
                       of \textbf{more than 11 million people}. Toxins produced
                       by harmful algal blooms  deeply affected the economy
                       and health of the environment and the public. %Coastal towns
                       %that rely on tourism are negatively affected by toxic algal
                      % blooms.
                  \begin{wrapfigure}{r}{0.45\textwidth}
                    \centering
                        \includegraphics[scale = 0.12]{assets/algalbloom.jpg}
                        \caption{Great Lakes: October 9, 2011}
                  \end{wrapfigure}

                \begin{itemize}
               		\item Drinking water is polluted.
					\item Local residents and visitors are prevented from boating, swimming,
          and visiting Lake Erie shorelines.
					%\item Nearby residents are vulnerable to illnesses caused by the toxins.
					\item Toxins can result in the death of marine life, and severely impact
          an aquatic ecosystem.
				\end{itemize}
                    \vspace*{3mm}
                \end{block}
                %%%%%%%%%%%%%%%%%%%%%%%%%%%%%%%%%%%
                %
                % Design
                %
                %%%%%%%%%%%%%%%%%%%%%%%%%%%%%%%%%%%
                \begin{alertblock}{\textsc{\textbf{System Design}}}
					   \begin{figure}
                    		\includegraphics[scale = 3.5]{assets/diagram.jpg}
                    	\caption{Different portions comprising the system}
                    	\end{figure}
                \end{alertblock}

            \end{column}
            %%%%%%%%%%%%%%%%%%%%%%%%%%%%%%%%%%%%%%%%%%%%%%%%%%%%%%%%%%%%%%%%%%%%%
            %
            % Center column 
            %
            %%%%%%%%%%%%%%%%%%%%%%%%%%%%%%%%%%%%%%%%%%%%%%%%%%%%%%%%%%%%%%%%%%%%%
            \begin{column}{.33\linewidth}
                 %%%%%%%%%%%%%%%%%%%%%%%%%%%%%%%%%%%
                %
                % Sensors
                %
                %%%%%%%%%%%%%%%%%%%%%%%%%%%%%%%%%%%
                \begin{block}{\textsc{\textbf{Sensors}}}
                    \vspace*{3mm}

                   In our system we used a collection of sensors, Arduino boards, and programs.
                    \begin{itemize}
                    	\item pH, conductivity, temperature, dissolved oxygen sensors
                      were used.
                    	\item These sensors are all connected to an Arduino board
                      for automatic data collection and analysis.
                     	\item Waterproofed sensors on this robotic system allows for
                      data to be collected for several hours at a time, which is
                      then transmitted to the analytics software.
                     \end{itemize}

                    \vspace*{3mm}
                \end{block}
                  %%%%%%%%%%%%%%%%%%%%%%%%%%%%%%%%%%%
                %
                % Robot
                %
                %%%%%%%%%%%%%%%%%%%%%%%%%%%%%%%%%%%
  				\begin{block}{\textsc{\textbf{Robotic and Sensor System}}}
                    \begin{itemize}
                    			\item A \textbf{waterproofed case} was designed to house sensors and boards on the robot.
							\item A remotely operated \textbf{robotic platform} was constructed with PVC pipe,
                      mesh, propellers, solder, and other materials, and is easily
                      deconstructable.
				 \end{itemize}
				
                    \begin{center}
                    \begin{figure}
                    \begin{tabular}{cc}
                    \includegraphics[scale = 0.15]{assets/IMG_9097.JPG}
                    \hspace*{5mm}
                    &
                    \includegraphics[scale = 0.42]{assets/meworking1}
                    \end{tabular}
                    \caption{Soldering sensor wires \hspace{30mm} Drilling holes for sensors}
                    \end{figure}
                    \end{center}

  				\end{block}

            \end{column}
            %%%%%%%%%%%%%%%%%%%%%%%%%%%%%%%%%%%%%%%%%%%%%%%%%%%%%%%%%%%%%%%%%%%%%
            %
            % Right column - Outcomes
            %
            %%%%%%%%%%%%%%%%%%%%%%%%%%%%%%%%%%%%%%%%%%%%%%%%%%%%%%%%%%%%%%%%%%%%%
            \begin{column}{.33\linewidth}

                %%%%%%%%%%%%%%%%%%%%%%%%%%%%%%%%%%%
                %
                % Testing
                %
                %%%%%%%%%%%%%%%%%%%%%%%%%%%%%%%%%%%
                \begin{block}{\textsc{\textbf{Testing}}}
                    \vspace*{3mm}

                    To complete our sensor system we used a collection of sensors,
                    Arduino boards, and programs.
                    \begin{itemize}
                    	\item The robotic system was initially tested in the pool for buoyancy and general operation validation.
                    	\item All sensors were calibrated and also tested individually.
                    	\begin{center}
                    	\begin{tabular}{l|l}
                    	\textbf{Sensors} & \textbf{Testing Values} \\
                    	\hline 
                    	Temperature & 40F - 70F \\
                    	pH & Standard Buffer Solutions 4.0 and 7.0 \\
                    	Dissolved Oxygen & 0.5 mol/L NaOH Solution \\
                    								&  (Sodium Hydroxide) \\
                    	Conductivity & Standard Buffer Solutions \\
                    								& 1413us/cm and 12.88ms/cm 
                    	\end{tabular}
                    	\end{center}
                     \end{itemize}

                    \vspace*{3mm}
                \end{block}

                %%%%%%%%%%%%%%%%%%%%%%%%%%%%%%%%%%%
                %
                % Results
                %
                %%%%%%%%%%%%%%%%%%%%%%%%%%%%%%%%%%%
                \begin{alertblock}{\textsc{\textbf{Testing and Future Work}}}
                    \vspace*{3mm}
                    \begin{itemize}
                    	\item First, software and the sensor data collection,
                      management and analysis were evaluated.
                    	\item Then, completed robotic system was tested in the pool
                      water.
                    	\item In June, measurements from Lake Erie will be taken at
                      different levels of the water column.
                    	\item The results of this project, including the collected
                      data and its analysis, will be shared with other researchers
                      and will be used to help students understand variations in
                      water quality measurements.
                    	%\item The data retrieved can be used to protect coastal environments and surrounding communities.
                    \end{itemize}

                    \vspace*{3mm}
                \end{alertblock}
            \end{column}

        \end{columns}
    \end{frame}
\end{document}
